\chapter{Introduzione}
\label{cap:introduzione}
La postura instabile e le cadute sono seri problemi di salute che affliggono una grande percentuale della popolazione\cite{rubenstein}. Molti casi di cadute colpiscono gli adulti con un età superiore ai 65 anni. Anche se molte non sono gravi, alcune sono seguite da complicazioni, quali fratture o lacerazioni e richiedono trattamenti medici e persino l'ospedalizzazione portando ad una diminuzione della mobilità con una conseguente perdita d'indipendenza che, a lungo termine, si traduce in una diminuzione della fiducia in se stessi del paziente e nel declino del benessere psicologico e motorio; spesso questi fattori sono collegati al calo della qualità della vita e ad una conseguente riduzione dell'aspettativa di vita del paziente\cite{krupitzer}.

Ricerche svolte in precedenza hanno rivelato che le cadute e le lesioni correlate possono essere prevenute studiando specifici fattori di rischio tramite l'analisi clinica della postura\cite{gillespie}. Da tempo è stato dimostrato che l'analisi posturografica è un ottimo metodo di calcolo degli indicatori di rischio di caduta correlando le cadute all'indebolimento della postura. Purtroppo molto spesso questo tipo di analisi viene evitata in favore di test più rapidi, che non richiedono strumenti costosi o personale esperto (come le pedane stabilometriche)\cite{mancini}.

Al giorno d'oggi è sempre più richiesta la presenza di strumenti per la misurazione della postura in maniera rapida, poco costosa e in grado di essere operati da utenti inesperti. Vista la sempre più crescente popolarità degli smartphone anche tra le persone più adulte è logico pensare a questi dispositivi come una buona opportunità per creare uno strumento in grado di risolvere questi bisogni\cite{roeing}. 

In questa tesi si propone la creazione di un'applicazione per smartphone in grado di analizzare la postura di un paziente in maniera autonoma, rapida ed affidabile. L'applicazione sviluppata prende il nome di {\bfseries Balance} ed è disponibile su piattaforme Android ed IOS. Nei capitoli seguenti si parlerà in maniera più approfondita di quali sensori presenti negli smartphone possono essere utilizzati per la stima della postura, quali parametri si estraggono dai dati dei sensori e di quali tecnologie e tecniche sono state impiegate per la creazione dell'applicazione.