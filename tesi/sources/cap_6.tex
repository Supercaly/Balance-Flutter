\chapter{Conclusioni}
\label{cap:conclusioni}
In questa tesi è stata sviluppata un'applicazione multipiattaforma per smartphone in grado di valutare la postura dell'utente raccogliendo dati dai sensori presenti all'interno del dispositivo. Immaginando il corpo umano come un singolo pendolo inverso è stato possibile convertire la variazione dei valori misurati dall'accelerometro in variazioni del centro di gravità ($COG_v$), in questo modo non è stato complesso ricavare i grafici di stabilogramma e statokinesigramma; dalla variazione del centro di gravità, inoltre, sono estratti i valori delle features studiando il segnale nel dominio del tempo e in quello della frequenza.

Come detto in precedenza lo sviluppo per dispositivi IOS non è stato approfondito perciò in futuro potrebbe risultare interessante terminare il lavoro anche su questa piattaforma.

Fra i potenziali sviluppi futuri dell'applicazione sicuramente è presente l'integrazione di un database on-line in cui poter raccogliere le misurazione di ogni utente assieme ai dati d'anamnesi e ai valori calcolati. In questo ambito risulta di particolare interesse il rispetto delle nuove norme sulla privacy e il GDPR che obbliga gli sviluppatori a prestare più attenzione a quali dati vengono richiesti agli utenti e al modo in cui sono salvati. Per esempio ogni utente che utilizza il servizio può essere rappresentato solo da un identificativo generato casualmente ed ogni informazione fa riferimento solo a quello, per garantire il totale anonimato. Un altro ambito di progresso interessante può essere l'utilizzo di una struttura di back-end composta da micro-servizi con sistemi di backup dei dati automatici.

Infine un'ultima area di miglioramento è la richiesta dei dati d'anamnesi; per questa è opportuno uno studio più approfondito, magari, consultando il parere di vari esperti in modo da delineare un gruppo maggiore di informazioni che possono essere correlate a problemi posturali. In questo ambito esistono numerose opportunità di ricerca svolgendo un lavoro di correlazione fra tutti i dati richiesti e i valori estratti dalle misurazioni utilizzando persino nuove tecnologie come ad esempio il machine learning.